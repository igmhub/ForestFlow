\author{
J.~Chaves-Montero\inst{\ref{inst0}}\thanks{jchaves@ifae.es}
\and
L.~Cabayol-Garcia\inst{\ref{inst0},\ref{inst1}}\thanks{lcabayol@pic.es}
\and
M.~Lokken\inst{\ref{inst0}}
\and
A.~Font-Ribera\inst{\ref{inst0}}\thanks{afont@ifae.es}
\and
J.~Aguilar\inst{\ref{inst2}}
\and
S.~Ahlen\inst{\ref{inst3}}
\and
D.~Bianchi\inst{\ref{inst19}}
\and
D.~Brooks\inst{\ref{inst23}}
\and
T.~Claybaugh\inst{\ref{inst2}}
\and
S.~Cole\inst{\ref{inst6}}
\and
A.~de la Macorra\inst{\ref{inst37}}
\and
S.~Ferraro\inst{\ref{inst2},\ref{inst43}}
\and
J.~E.~Forero-Romero\inst{\ref{inst52},\ref{inst53}}
\and
E.~Gaztañaga\inst{\ref{inst55},\ref{inst27}}
\and
S.~Gontcho A Gontcho\inst{\ref{inst2}}
\and
G.~Gutierrez\inst{\ref{inst25}}
\and
K.~Honscheid\inst{\ref{inst33},\ref{inst46},\ref{inst47}}
\and
R.~Kehoe\inst{\ref{inst66}}
\and
D.~Kirkby\inst{\ref{inst14}}
\and
A.~Kremin\inst{\ref{inst2}}
\and
A.~Lambert\inst{\ref{inst2}}
\and
M.~Landriau\inst{\ref{inst2}}
\and
M.~Manera\inst{\ref{inst75},\ref{inst0}}
\and
P.~Martini\inst{\ref{inst33},\ref{inst65},\ref{inst47}}
\and
R.~Miquel\inst{\ref{inst77},\ref{inst0}}
\and
A.~Muñoz-Gutiérrez\inst{\ref{inst37}}
\and
G.~Niz\inst{\ref{inst36},\ref{inst11}}
\and
I.~P\'erez-R\`afols\inst{\ref{inst84}}
\and
G.~Rossi\inst{\ref{inst88}}
\and
E.~Sanchez\inst{\ref{inst38}}
\and
M.~Schubnell\inst{\ref{inst7}}
\and
D.~Sprayberry\inst{\ref{inst21}}
\and
G.~Tarl\'{e}\inst{\ref{inst7}}
\and
B.~A.~Weaver\inst{\ref{inst21}}
}

J. Chaves-Montero, L. Cabayol-Garcia, M. Lokken, A. Font-Ribera, J. Aguilar, S. Ahlen, D. Bianchi, D. Brooks, T. Claybaugh, S. Cole, A. de la Macorra, S. Ferraro, J. E. Forero-Romero, E. Gaztañaga, S. Gontcho A Gontcho, G. Gutierrez, K. Honscheid, R. Kehoe, D. Kirkby, A. Kremin, A. Lambert, M. Landriau, M. Manera, P. Martini, R. Miquel, A. Muñoz-Gutiérrez, G. Niz, I. Pérez-Ràfols, G. Rossi, E. Sanchez, M. Schubnell, D. Sprayberry, G. Tarlé, and B. A. Weaver

On large scales, measurements of the Lyman-$\alpha$ forest offer insights into the expansion history of the Universe, while on small scales, these impose strict constraints on the growth history, the nature of dark matter, and the sum of neutrino masses. This work introduces ForestFlow, a cosmological emulator designed to bridge the gap between large- and small-scale Lyman-$\alpha$ forest analyses. Using conditional normalizing flows, ForestFlow emulates the 2 Lyman-$\alpha$ linear biases ($b_\delta$ and $b_\eta$) and 6 parameters describing small-scale deviations of the 3D flux power spectrum ($P_\mathrm{3D}$) from linear theory. These 8 parameters are modeled as a function of cosmology $\unicode{x2013}$ the small-scale amplitude and slope of the linear power spectrum $\unicode{x2013}$ and the physics of the intergalactic medium. Thus, in combination with a Boltzmann solver, ForestFlow can predict $P_\mathrm{3D}$ on arbitrarily large (linear) scales and the 1D flux power spectrum ($P_\mathrm{1D}$) $\unicode{x2013}$ the primary observable for small-scale analyses $\unicode{x2013}$ without the need for interpolation or extrapolation. Consequently, ForestFlow enables for the first time multiscale analyses. Trained on a suite of 30 fixed-and-paired cosmological hydrodynamical simulations spanning redshifts from $z=2$ to $4.5$, ForestFlow achieves $3$ and $1.5\%$ precision in describing $P_\mathrm{3D}$ and $P_\mathrm{1D}$ from linear scales to $k=5\,\mathrm{Mpc}^{-1}$ and $k_\parallel=4\,\mathrm{Mpc}^{-1}$, respectively. Thanks to its parameterization, the precision of the emulator is also similar for both ionization histories and two extensions to the $\Lambda$CDM model $\unicode{x2013}$ massive neutrinos and curvature $\unicode{x2013}$ not included in the training set. ForestFlow will be crucial for the cosmological analysis of Lyman-$\alpha$ forest measurements from the DESI survey.